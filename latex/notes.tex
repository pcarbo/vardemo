\documentclass[final]{siamltex}
\usepackage{amsmath}
\usepackage{amssymb}
\usepackage{bm}

\setlength{\oddsidemargin}{0.65in}
\setlength{\evensidemargin}{0.65in}

\newcommand{\smfrac}[2]{{\textstyle\frac{#1}{#2}}}
\newcommand{\half}{{\textstyle\frac{1}{2}}}
\newcommand{\smint}{\textstyle\int}
\newcommand{\logodds}{\mbox{log-odds}}
\newcommand{\lb}{\bm{[}}
\newcommand{\rb}{\bm{]}}

\title{Laplace's method and a variational approximation for a small
  logistic regression example}

\author{Peter Carbonetto\thanks{Dept. of Human Genetics, 
University of Chicago, 920 E58th St., 4th floor, Chicago, IL, 60637}}

\begin{document}

\maketitle

\section*{Summary}

The goal of these lecture notes is to develop a practical approach to
implement hypothesis testing with logistic regression. We're going to
look at two different approaches to this problem: using Laplace's
method, and using variational approximation techniques.

\section*{Logistic regression}

To make things concrete as possible, suppose we want to assess the
ability of our variable ($X$) to predict, or explain, disease ($Y$). I
denote sample $i$ in our data set by $x_i$ and $y_i$. The logistic
model says that the logarithm of the odds of disease, denoted by
$\logodds(i)$, is a linear combination of $x_i$:
\begin{align*}
\logodds(i) = \log\bigg\{\frac{p(y_i = 1 \,|\, x_i, \beta)}
         {p(y_i = 0 \,|\, x_i, \beta)}\bigg\} 
= \beta_0 + x_i \beta.
\end{align*}
I use $\pi_i$ as shorthand for
$p(y_i = 1 \,|\, x_i, \beta)$, and write the summation above using the
dot product $x_i^T\beta$, so the logistic model is
\begin{align*}
\logodds(i) = \log\bigg\{\frac{\pi_i}{1-\pi_i}\bigg\} = \beta_0 + x_i\beta.
\end{align*}
In other words, the disease status is a coin toss with success rate
\begin{align*}
\pi_i = \sigma(x_i\beta),
\end{align*}
where $\sigma(x) = 1/(1 + e^{-x})$ is the sigmoid function (it is the
inverse of the logit function).

Assuming independence of the samples, the likelihood of $y = (y_1,
\ldots, y_n)^T$ given $x = (x_1, \ldots, y_n)^T$ and $\beta$ is
\begin{align*}
p(y \,|\, x, \beta) = \prod_{i=1}^n \pi_i^{y_i} (1-\pi_i)^{1-y_i}.
\end{align*}

One reason that the logistic model is so popular is because it is easy
to compute the maximum likelihood or {\em maximum a posteriori}
estimator. Instead, I'm going to focus on the problem of computing the
ratio of the marginal likelihoods,
\begin{align*}
\frac{p(y \,|\, x, H_1)}
     {p(y \,|\, x, H_0)} = 
\frac{\int p(y \,|\, x, \beta) \, p(\beta) \, d\beta}
     {p(y \,|\, x, H_0)},
\end{align*}
where $H_0$ is, for the sake of illustration, the null hypothesis that
no factor contributes to disease risk, and $H_1$ is alternative
hypothesis in which $X$ increases or decreases susceptibility to
disease. To compute this ratio of likelihoods---what we call the Bayes
factor---we need to marginalize or integrate out the random vector
$\beta$. This integral has no closed-form solution, so we must compute
it {\em numerically}.

There is, of course, the option of using Monte Carlo methods, but I'm
going to focus on a couple alternative approaches.

\section*{Laplace's method}

A simple and general-purpose method we can use to approximate the
marginal likelihood is Laplace's method, which is nothing more than a
Taylor series approximation to the logarithm of the density
function. Let's look at Laplace's method first in the general case.

For some nonlinear function $f(x)$, with $x \in \mathbb{R}^d$, suppose
we want to compute the integral
\begin{align*}
I = \smint e^{f(x)} \, dx
\end{align*}
that has no known closed-form solution. (We'll write our marginal
likelihood in this form.) First, we form a second-order Taylor series
approximation about point $\hat{x}$:
\begin{align*}
f(x) \approx f(\hat{x}) + g^T(x - \hat{x}) + \half(x - \hat{x})^TH(x-\hat{x}),
\end{align*}
where $g = \nabla f(\hat{x})$ is the vector-valued function of
first-order partial derivatives (the gradient) at $\hat{x}$, and $H =
\nabla^2 f(\hat{x})$ is the matrix of second-order partial derivatives
(the Hessian) at $\hat{x}$. This will be a good a approximation near
$\hat{x}$, but will get increasingly worse as we get further away from
$\hat{x}$.

The second-order Taylor-series approximation to the integral is
\begin{align*}
I \approx \smint 
e^{f(\hat{x}) + g^T(x - \hat{x}) + \half(x - \hat{x})^TH(x-\hat{x})} \, dx.
\end{align*}
To solve this integral, we'll use the fact that the multivariate
normal density with mean $\mu$ and covariance $\Sigma$ integrates to
one:
\begin{align*}
\smint |2\pi\Sigma|^{-1/2} e^{-\frac{1}{2}(x-\mu)^T\Sigma^{-1}(x-\mu)} 
\, dx = 1.
\end{align*}
Here, $|A|$ denotes the determinant of matrix $A$. Next, making the
substitutions $\Sigma = -H^{-1}$ and $\mu = \Sigma g + \hat{x}$, and
rearranging terms, we find that
\begin{align*}
I \approx |2\pi\Sigma|^{1/2} 
e^{f(\hat{x}) + \frac{1}{2}(\hat{x}-\mu)^T\Sigma^{-1}(\hat{x}-\mu)}.
\end{align*}

Three remarks here. The first remark is that the covariance is equal
to the negative of the inverse of the Hessian. Since, the covariance
must be positive definite, the Hessian must be negative definite at
$\hat{x}$. If $f(x)$ is a concave function, then the Hessian is
guaranteed to be negative definite. This is the case for our logistic
model.

Second, observe that this is general procedure; nowhere did we require
that the function $f(x)$ correspond to a probability density function.

Third, if $f(x)$ corresponds to some density function, then what we
have effectively done is replaced this density that is difficult to
integrate with a normal density with mean $\mu$ and covariance
$\Sigma$.

Before applying this to our problem, let me point out two special
cases of Laplace's method. One way to improve the approximation is to
first find a maximum of the function, so that the approximation is
best where there is the most support. If $\hat{x}$ is a local maximum,
it follows that $g = 0$, so the mean is simply $\mu = \hat{x}$, and
the Laplace approximation reduces to
\begin{align*}
I \approx |2\pi\Sigma|^{1/2} e^{f(\hat{x})}.
\end{align*}

The second interesting case occurs when we center our quadratic
approximation at $\hat{x} = 0$. This might be a sensible course of
action if we know that most of the risk factors will be zero (such as
in a genome-wide association study). In this case,
\begin{align*}
I &\approx |2\pi\Sigma|^{1/2}
e^{f(\hat{x}) + \frac{1}{2}\mu^T\Sigma^{-1}\mu}.
\end{align*}

Now let's apply this result to our marginal likelihood of
interest. First, I need to rewrite our integral in the form
\begin{align*}
\smint p(y \,|\, X, \beta) \, p(\beta) \, d\beta
= C \smint e^{f(\beta)} \, d\beta.
\end{align*}
The function $f(\beta)$ plays the role of the log-density of the
likelihood and the prior; $e^{f(\beta)}$ is the posterior up to a
normalizing constant. Assuming the prior is normal with mean zero and
variance $\sigma_0^2$, the likelihood times the prior is
\begin{align*}
p(y \,|\, X, \beta) \, p(\beta) &= (2\pi\sigma_0^2)^{-1/2} e^{f(\beta)},
\end{align*}
where
\begin{align*}
f(\beta) = \sum_{i=1}^n y_i\log\pi_i + \sum_{i=1}^n (1-y_i)\log(1-\pi_i)
- \beta^2/(2\sigma_0^2).
\end{align*}
Recall that $\beta$ appears in the $\pi_i$'s.

Notice that we can rewrite the log-density $f(\beta)$ as
\begin{align*}
f(\beta) &= \sum_{i=1}^n (y_i-1)\logodds(i) + \sum_{i=1}^n \log\pi_i
-\beta^2/(2\sigma_0^2).
\end{align*}
Here I've divided the terms in the exponent into two parts: those that
are linear or quadratic in $\beta$, and those that are nonlinear in
$\beta$. The only part that requires approximation is the nonlinear
part.

Let's now investigate the first strategies I mentioned earlier: set
$\hat{\beta}$ to the maximum of $f(\beta)$. To derive the result I'll
use the fact that the gradient and Hessian of the log-density ({\em
  i.e.} the first and second-order derivatives with respect to
$\beta$) work out to be
\begin{align*}
\nabla f(\beta)   &= x^T(y-\pi) - \beta/\sigma_0^2 \\
\nabla^2 f(\beta) &= -1/\sigma_0^2 - x^TWx,
\end{align*}
where $\pi$ is the vector with entries $\pi_i$, and $W$ is the matrix
with diagonal entries $\pi_i(1-\pi_i)$. The expression for the
gradient makes sense: ignoring the prior, we obtain a maximum, roughly
speaking, when the predictions $\pi_i$ match the observed labels
$y_i$. For $\hat{\beta} = \beta^{\mathrm{(MAP)}}$, the final result is
easy to calculate:
\begin{align*}
I \approx \sigma/\sigma_0 e^{f(\hat{\beta})},
\end{align*} 
with variance $\sigma^2 = (1/\sigma_0^2 + x^TWx)^{-1}$.

\section*{Variational method}

The first step in developing our variational inference procedure is to
deal with the nonlinear $\log\pi_i$ terms in our log-density function
$f(\beta)$. The basic idea is to formulate a lower bound to the
logarithm of the sigmoid function. Skipping the technical details (see
Christopher Bishop's book), we obtain the lower bound
\begin{align*}
\log\sigma(x) \geq 
\log\sigma(\theta) + \half(x-\theta) - \smfrac{u}{2}(x^2-\theta^2),
\end{align*}
where $u = \frac{1}{\theta}(\sigma(\theta) - \half)$, and $\theta \geq
0$ adjusts this lower bound. I will have one parameter $\theta$ for
every sigmoid term or sample. Notice that the terms involving
$x$---which will later be replaced by linear combinations of
$\beta$---are conveniently linear or quadratic. This leads directly to
a lower bound on the log-likelihood $\log p(y \,|\, x, \beta)$, which
I denote by $L(\beta; \theta)$:
\begin{align*}
L(\beta; \theta) &= 
(y-1)^T\logodds(i) + \sum_{i=1}^n \log\sigma(\theta_i) 
+ \half(\logodds(i) - \theta_i) 
- \smfrac{u_i}{2}(\logodds(i)^2 - \theta_i^2).
\end{align*}
When there is no intercept ($\beta_0 = 0$), this simplifies to
\begin{align*}
L(\beta; \theta) &= \sum_{i=1}^n \log\sigma(\theta_i) +
\smfrac{\theta_i}{2}(u_i\theta_i - 1)
+ (y-\half)^Tx\beta - x^TUx\beta^2/2.
\end{align*}
By extension, we obtain a lower bound on the marginal likelihood:
\begin{align*}
I 
= \smint p(y \,|\, x, \beta) \, p(\beta) \, d\beta 
= \smint p(\beta) \, e^{f(\beta)} \, d\beta 
\geq \smint p(\beta) \, e^{L(\beta; \theta)} \, d\beta.
\end{align*}
Here, $U$ is the $n \times n$ matrix with diagonal entries $u_i$.
Compare the lower bound $L(\beta; \theta)$ to the likelihood for a
linear regression with coefficients $\beta$: $(y - \half)$ is a vector
of measurements for a surrogate quantitative trait, and $U$ scales the
inverse of the covariance matrix.

{\bf Note:} for the remainder, I'll assume $\beta_0 = 0$, just to make
these expressions a little bit simpler. The MATLAB code allows for
arbitrary intercept $\beta_0$.

Now that we've derived a lower bound on $I$, there are two pressing
questions: How do we compute this lower bound? And how do we adjust
the parameters $\theta = (\theta_1, \ldots, \theta_n)$ so that the
lower bound is as tight as possible? To resolve the first question,
suppose that the prior is normal with zero mean and variance
$\sigma_0^2$. Then $p(\beta) \, e^{L(\beta; \theta)}$ is, up to a
constant of proportionality, multivariate normal with mean $\mu =
\sigma^2 x^T(y - \half)$ and variance $\sigma^2 = (1/\sigma_0^2 +
x^TUx)^{-1}$. In this special case, the expression for the lower bound
works out to be
\begin{align*}
I \geq \sigma/\sigma_0
\exp\Big\{ \half \mu^2/\sigma^2 + 
{\textstyle \sum_{i=1}^n \log\sigma(\theta_i) + 
\smfrac{\theta_i}{2}(u_i\theta_i - 1)}\Big\}.
\end{align*}

The second problem is perhaps most easily resolved by interpreting
within the EM framework:in the E-step, compute expectations (the mean
and covariance) of the unknowns $\beta$; and in the M-step, compute
the {\em maximum a posteriori} estimator of $\theta$, which amounts to
maximizing the expected value of $L(\beta; \theta)$. (We can ignore
the $\log p(\beta)$ term because $\theta$ does not affect the prior.)

To derive the M-step, first take the partial derivatives of the
expected log-likelihood with respect to the variational parameters:
\begin{align*}
\frac{\partial E\lb L(\beta;\theta)\rb}{\partial \theta_i} &=
\frac{u_i'}{2}(\theta_i^2 - (x_i\mu)^2 - x_i^2\sigma^2),
\end{align*}
where $\mu$ is the posterior mean and $\Sigma$ is the posterior
covariance computed in the E-step. The usual procedure is to set the
partial derivatives to zero and solve for $\theta$. At first glance,
this does not appear to be possible. But a couple of observations will
yield a closed-form solution: first, the slope $u$ is symmetric in
$\theta$, so we only need to worry about the positive quadrant;
second, for $\theta > 0$, $u$ is strictly monotonic as a function of
$\theta$, and so $u'$ is never zero. Therefore, we can solve for the
fixed point:
\begin{align*}
\theta_i = \sqrt{(x_i^T\mu)^2 + x_i^T \Sigma x_i}.
\end{align*}
This EM algorithm for adjusting the variational parameters $\theta$
has an intuitive appeal: we are adjusting the lower bound so that the
approximation is tighest where there is most support for $\beta$.
  
\end{document}
